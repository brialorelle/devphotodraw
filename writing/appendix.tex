\clearpage
\makeatletter
\efloat@restorefloats
\makeatother


\begin{appendix}
\hypertarget{tracing-scores}{%
\section{Tracing Scores}\label{tracing-scores}}

To compute these error components, we applied an image registration
algorithm, AirLab (Sandkühler, Jud, Andermatt, \& Cattin, 2018), to
align each tracing to the target shape, yielding an affine
transformation matrix that minimized the pixel-wise correlation distance
between the aligned tracing, \(T\), and the target shape, \(S\):
\(Loss_{NCC} = - \frac{\sum S \cdot T - \sum E(S) E(T)}{N \sum Var(S) Var(T)}\),
where \(N\) is the number of pixels in both images. The shape error was
defined by the final correlation distance between the aligned tracing
and the target shape. The spatial error was defined by the magnitude of
three distinct error terms: location, orientation, and size error,
derived by decomposing the affine transformation matrix above into
translation, rotation, and scaling components, respectively. In sum,
this procedure yielded four error values for each tracing: one value
representing the shape error (i.e., the pixel-wise correlation distance)
and three values representing the spatial error (i.e., magnitude of
translation, rotation, scaling components).

We used the tracing quality ratings to obtained in Long, Fan, Chai, \&
Frank (2021) to assign weights to each of their error terms; adult
observers (\(N\)=70) rated 1325 tracings (i.e., 50-80 tracings per shape
per age) and evaluated ``how well the tracing matches the target shape
and is aligned to the position of the target shape'' on a 5-point scale.
An ordinal regression mixed-effects model to predict these 5-point
ratings, which contained correlation distance, translation, rotation,
scaling, and shape identity (square vs.~star) as predictors, with random
intercepts for rater. This model yielded parameter estimates that could
then be used to score each tracing in the dataset; we averaged scores
for both shapes to yield a single tracing score for each participant.

\hypertarget{refs}{}
\leavevmode\hypertarget{ref-long2021parallel}{}%
Long, B., Fan, J., Chai, Z., \& Frank, M. C. (2021). Parallel
developmental changes in children's drawing and recognition of visual
concepts.

\leavevmode\hypertarget{ref-sandkuhler2018}{}%
Sandkühler, R., Jud, C., Andermatt, S., \& Cattin, P. C. (2018). AirLab:
Autograd image registration laboratory. \emph{arXiv Preprint
arXiv:1806.09907}.
\end{appendix}
